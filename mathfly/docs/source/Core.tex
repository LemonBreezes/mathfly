\documentclass[12pt]{article}
\usepackage[T1]{fontenc}
\usepackage{longtable}
\title{Mathfly core commands}
\author{Mike Roberts}
\begin{document}
\maketitle
\tableofcontents
\section{Introduction}
These commands are enabled with the voice command "enable core" and are common across all editors. They can be modified by editing mathfly/config/core.toml in any text editor or saying "configure Core" while the module is enabled.

\section{Phonetic alphabet}
By default the NATO phonetic alphabet is used. Letters can be prefixed with "big" to get capitals.

\begin{longtable}{ l l}
\noindent
alpha & a \\
bravo & b \\
charlie & c \\
delta & d \\
echo & e \\
foxtrot & f \\
golf & g \\
hotel & h \\
india & i \\
juliet & j \\
kilo & k \\
lima & l \\
mike & m \\
november & n \\
oscar & o \\
papa & p \\
quebec & q \\
romeo & r \\
sierra & s \\
tango & t \\
uniform & u \\
victor & v \\
whiskey & w \\
x-ray & x \\
yankee & y \\
zulu & z \\
\end{longtable}

\section{Punctuation}
\begin{longtable}{ l l}
ampersand & \& \\
apostrophe & ' \\
(asterisk | starling) & * \\
at sign & @ \\
backslash & \textbackslash \\
backtick & ` \\
pipe symbol & | \\
caret & \textasciicircum \\
colon & : \\
comma & , \\
dollar & \$ \\
(dot | point) & . \\
(quote | quotes) & " \\
equals & = \\
exclamation & ! \\
hashtag & \# \\
hyphen & - \\
minus & - \\
percent & \% \\
plus & + \\
question mark & ? \\
semicolon & ; \\
slash & / \\
single quote & ' \\
tilde & \textasciitilde \\
underscore & \textunderscore \\
greater than & $>$ \\
less than & $<$ \\
brax & [~] \\
prekris & (~) \\
curly & \{~\} \\
\end{longtable}

\section{Repeatable keys}
These keys may all be repeated up to 10 times by saying a number afterwards, e.g. "shock two" will press enter twice.

\noindent
\begin{longtable}{ l l}
(space | ace) & space \\
(tab | tabby) & tab \\
(enter | shock) & enter \\
check & ctrl-enter \\
(backspace | clear) & backspace \\
(delete | deli) & delete \\
splat & delete words \\
\end{longtable}

\section{Other keys}
\begin{longtable}{ l l}
(escape | eskimo) & escape \\
shackle & selects the current line \\
(copy | stoosh) & copy the current selection \\
(paste | spark) & paste from clipboard \\
cut & cut the current selection \\
(duple | duplicate) & duplicate the current line \\
home & home \\
end & end \\
save & save the current file \\
\end{longtable}

\section{Movement and selection}
All of these commands are repeatable by saying a number afterwards.

\subsection{Directions}
\begin{longtable}{ l l}
left & left \\
right & right \\
up & up \\
down & down \\
\end{longtable}

\subsection{Modifiers}
Saying one of these before a direction will change the commands behaviour.

\noindent
\begin{longtable}{ l l}
shin & select by character \\
queue & select by word \\
fly & move by word \\
\end{longtable}

Saying "wally" after a movement command will use the home/end keys instead of directions.

\subsection{Examples}
\begin{longtable}{ l l}
left three & move left three \\
left wally & go to the beginning of the line \\
down wally & go to the bottom of the document \\
queue right four & select four words to the right \\
shin right wally & select until the end of the line \\
\end{longtable}

\section{Aliases}
Aliases allow for easy on-the-fly creation of new commands. Simply highlight a piece of text or mathematics (using eg ``queue left two'') and say the command "alias" followed by a memorable command name. From now on, saying that command name will paste in the highlighted text. To delete aliases, use the command ``delete aliases''. Does not work in Scientific Notebook at the moment.
\end{document}