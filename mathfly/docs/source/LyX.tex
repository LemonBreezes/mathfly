\documentclass[12pt]{article}
\usepackage{tabularx}
\usepackage{longtable}
\usepackage{amssymb}
\usepackage[T1]{fontenc}
\usepackage{geometry}
\title{Dictating mathematics into LyX using Mathfly}
\author{Mike Roberts}
\begin{document}
\maketitle
\tableofcontents

\section{Introduction}
\begin{itemize}
\item \textbf{All of these bindings can be easily changed by modifying mathfly/config/lyx.toml in any text editor or saying "configure LyX" while the module is enabled.}
\item (option a | option b) means that both commands will do the same thing.
\item Square brackets means that the word(s) inside are optional, the command will work with or without them.
\end{itemize}

\section{Basics}
\begin{tabularx}{\linewidth}{ l X}
new file & Create a new file \\
open file & Open a file \\
save as & Save as \\
math mode & Insert in-line mathematics \\
display mode & Insert equation \\
normal mode & Insert regular text \\
next tab [<n>] & Navigate to next tab n times \\
previous tab [<n>] & Navigate to previous tab n times \\
close tab [<n>] & Close the current tab n times \\
view PDF & View current document as a PDF \\
update PDF & Refresh changes \\
move line up [<n>] & Move the current line up \\
move line down [<n>] & Move the current line down \\
\hline
insert [bulleted] list & Insert a list \\
insert numbered list & Insert a numbered list \\
insert description & Insert a description \\
insert part & Create a part label \\
insert (section | heading) & Insert a heading \\
insert sub (section | heading) & Insert a subheading \\
insert sub sub (section | heading) & Insert a sub subheading \\
insert paragraph & Insert a paragraph \\
insert sub paragraph & Insert a subparagraph \\
insert title & Provide a title \\
insert author & Provide an author \\
insert date & Provide a date \\
insert abstract & Insert an abstract \\
insert address & Insert an address \\
insert bibliography & Insert a bibliography \\
insert quotation & Insert a quotation \\
insert quote & Insert a quote \\
insert verse & Insert verse \\
insert delimiters & Insert more complex bracketry \\
insert matrix & Insert a matrix (see matrix section for more options) \\
insert macro & Create a new macro \\
\end{tabularx}

\section{Miscellaneous}
\begin{tabularx}{\linewidth}{ l X}
math mode & Begins a new mathematical dictation environment, necessary for all maths dictation. \\
new math line & Begins a new mathematical dictation line. \\
fraction & Creates a fraction. anything highlighted will form the numerator. \\
over & Creates a fraction with the previous element as the numerator (e.g. "five over three") \\
(super script | to the power) & Superscript \\
sub script & Subscript \\
squared & Superscript 2 \\
cubed & Superscript 3 \\
inverse & Superscript -1 \\
(parens | parentheses) & Parentheses \\
square brackets & Square brackets \\
curly brackets & Curly brackets \\
absolute & Create two bars and moves inside them \\
summation & $\sum^{a}_{b}$ \\
blank summation & $\sum$ \\
(summation | sum) to N & $\sum^{n}_{?}$ \\
product & $\prod^{a}_{b}$ \\
blank product & $\prod$ \\
product to N & $\prod^{n}_{?}$ \\
limit & $\lim_{?}$ \\
blank limit & $\lim$ \\
prime & $\prime$ (prime symbol) \\
degrees & $^{\circ}$ \\
exponential & $\exp ()$ \\
expectation & $E()$ \\
variance & $Var()$ \\
label above & Add a label above the selected text \\
label below & Add a label below the selected text \\
\end{tabularx}


\section{Letters}
\subsection{Greek}
By default, all of these commands must be prefixed with "greek" for lowercase or "greek big" for uppercase. This behaviour can be changed by modifying greek\_prefix and capitals\_prefix. Where relevant I have provided pronunciation tips for best results.


\

\begin{longtable}{llll}
alpha & $\alpha$ & & \\
beta & $\beta$ &  & beater \\
gamma & $\gamma$ & $\Gamma$ & \\
delta & $\delta$ & $\Delta$ & \\
epsilon & $\varepsilon$ & & \\
zeta & $\zeta$ & & \\
eta & $\eta$ & & eater \\
theta & $\theta$ & $\Theta$ & they-tah \\
iota & $\iota$ & & \\
kappa & $\kappa$ & & \\
lambda & $\lambda$ & $\Lambda$ & \\
mu & $\mu$ & & moo \\
nu & $\nu$ & & new \\
xi & $\xi$ & $\Xi$ & zee \\
pi & $\pi$ & $\Pi$ & \\
rho & $\rho$ & & \\
sigma & $\sigma$ & $\Sigma$ & \\
tau & $\tau$ & & \\
upsilon & $\upsilon$ & $\Upsilon$ & \\
phi & $\phi$ & $\Phi$ & \\
chi & $\chi$ & & kie \\
psi & $\psi$ & $\Psi$ & sigh \\
omega & $\omega$ & $\Omega$ & \\
\end{longtable}


\subsection{Accents}
These commands add accents above the highlighted text, or create an empty accent if nothing is highlighted.

\

\noindent
\begin{longtable}{ l l}
accent hat & $\hat{a}$ \\
accent tilde & $\tilde{a}$ \\
accent dot & $\dot{a}$ \\
accent double dot & $\ddot{a}$ \\
accent bar & $\bar{a}$ \\
accent vector & $\vec{a}$ \\
\end{longtable}


\section{Symbols}
In order to avoid clutter and misrecognition, mathematical symbols are split up into two distinct groups: common and uncommon. By default, common symbols (e.g. integral) need no prefix, while uncommon symbols (e.g. up arrow) are prefixed with "symbol". The prefixes are defined by symbol1\_prefix and symbol2\_prefix. It is expected that you will want to move symbols which you happen to use frequently or infrequently between the two groups, or change/remove the prefixes to your liking. There is a trade-off to be made between recognition accuracy and speed of dictation.

\subsection{Common symbols}
\noindent
\begin{longtable}{ l l}
[generic] root & $\sqrt[n]{x}$ \\
square root & $\sqrt{x}$ \\
integral & $\int$ \\
double integral & $\int \int$ \\
triple integral & $\int \int \int$ \\
times & $\times$ \\
divide & $\div$ \\
stop & $\cdot$ \\
plus or minus & $\pm$ \\
partial & $\partial$ \\
nice fraction & a/b \\
binomial & ${a \choose b}$ \\
infinity & $\infty$ \\
dot dot dot & $\dots$ \\
vector nabla & $\nabla$ \\
%relations
greater [than] [or] equal [to] & $\geq$ \\
less [than] [or] equal [to] & $\leq$ \\
not equal [to] & $\neq$ \\
approximately [equal] [to] & $\approx$ \\
proportional [to] & $\propto$ \\
preference less [than] & $\prec$ \\
preference less equal & $\preceq$ \\
preference greater [than] & $\succ$ \\
preference greater equal & $\succeq$ \\
% trig
sine & $\sin$ \\
cosine & $\cos$ \\
tangent & $\tan$ \\
secant & $\sec$ \\
cosecant & $\csc$ \\
cotangent & $\cot$ \\
arc sine & $\arcsin$ \\
arc cosine & $\arccos$ \\
arc tan & $\arctan$ \\
hyperbolic sine & $\sinh$ \\
hyperbolic cosine & $\cosh$ \\
hyperbolic tangent & $\tanh$ \\
hyperbolic cotangent & $\coth$ \\
% text
degree & $\deg$ \\
determinant & $\det$ \\
dimension & $\dim$ \\
(natural (log | logarithm) | log natural) & $\ln$ \\
logarithm & $\log$ \\
argument & $\arg$ \\
maximum & $\max$ \\
minimum & $\min$ \\
(modulo | modulus) & $\bmod$ \\
supremum & $\sup$ \\
infimum & $\inf$ \\
probability & $\Pr$ \\
%sets
there exists & $\exists$ \\
member [of] & $\in$ \\
for all & $\forall$ \\
empty set & $\emptyset$ \\
subset & $\subset$ \\
superset & $\supset$ \\
strict subset & $\subsetneq$ \\
strict superset & $\supsetneq$ \\
intersection & $\cap$ \\
union & $\cup$ \\
real numbers & $\mathbb{R}$ \\
complex numbers & $\mathbb{C}$ \\
integer numbers & $\mathbb{Z}$ \\
rational numbers & $\mathbb{Q}$ \\
natural numbers & $\mathbb{N}$ \\
logic and & $\land$ \\
logic or & $\lor$ \\
logic not & $\lnot$ \\
left arrow & $\leftarrow$ \\
right arrow & $\rightarrow$ \\
up arrow & $\uparrow$ \\
down arrow & $\downarrow$ \\
left right arrow & $\leftrightarrow$ \\
maps to & $\mapsto$ \\
oh plus & $\oplus$ \\
oh times & $\otimes$ \\
big oh plus & $\bigoplus$ \\
big oh times & $\bigotimes$ \\
diagonal dots & $\ddots$ \\
horizontal dots & $\cdots$ \\
vertical dots & $\vdots$ \\
\end{longtable}

\subsection{Less common symbols}
Prefix with "symbol"

\begin{longtable}{ll}
GCD & $\gcd$ \\
cat hom & $\hom$ \\
kernel & $\ker$ \\
% unit & $\unitone$ \\
% unit two & $\unittwo$ \\
% unit fraction  & $\unitfrac$ \\
% text fraction & $\tfrac$ \\
% display fraction & $\dfrac$ \\
% continued fraction & $\cfrac$ \\
% continued fraction (left) & $\cfracleft$ \\
% continued fraction (right) & $\cfracright$ \\
% text binomial & $\tbinom$ \\
% display binomial & $\dbinom$ \\
% strict subset & $\subsetneq$ \\
% strict superset & $\supsetneq$ \\
\end{longtable}

\section{Text modes}
These commands allow you to insert various forms of regular text into a mathematical environment. They should all be prefixed with "text".

\

\begin{longtable}{ l l}
(beebee|blackboard bold | blackboard) & $\mathbb{R N Z}$ \\
roman & $\mathrm{Sample text}$ \\
bold & $\mathbf{Sample text}$ \\
sans serif & $\mathsf{Sample text}$ \\
italic & $\mathit{Sample text}$ \\
typewriter & $\mathtt{Sample text}$ \\
\end{longtable}

\section{Fractions}
There are a few ways of easily inserting fractions:

\begin{itemize}
\item Use the "fraction" command, and navigate through it using directions.
\item Use the "over" command, which will build a fraction with the previous element as the numerator. e.g. "x-ray squared over five".
\item For denominators up to 10, use their natural names, providing a number for the numerator, e.g. "five thirds".
\end{itemize}

\section{Nested commands}
There are a few commands within Mathfly which allow for commands to be inserted within them. These are just examples, you can include any commands you want:
\begin{itemize}
\item ``Integral from minus infinity to infinity'' - integral symbol with superscript and subscript.
\item ``Definite from zero to ten'' - definite integral square brackets with subscript and superscript afterwards.
\item ``Differential x-ray squared by squared yankee'' - creates a differential friction.
\item ``Sum from india equals one to november'' - creates a summation.
\item ``Limit from november to infinity'' - create a limit.
\item ``argument that maximises greek beta'' - argmax.
\item ``minimum by greek beta'' - min.
\item ``sub india'' - quick sub/superscripts
\end{itemize}

\section{Matrices}
\begin{itemize}
\item To insert a matrix of a particular size, use the matrix command, e.g. "matrix three by one".
\item To add or remove columns and rows, Use the command "add/remove matrix column/row".
\item Matrices can be encased in brackets as expected, E.g. "parens matrix three by three".
\end{itemize}

\section{Environments}
These commands provide more detailed control over equation positioning and alignment.

\begin{tabularx}{\linewidth}{ l X}
insert (in line formula | in line) & In-line formula - same as "math mode" \\
insert numbered formula & Numbered formula \\
insert (display formula | display) & Same as "display mode" \\
insert (equation array environment | equation array) & Insert equation array - use the "check" (ctrl-enter) command to start a new line \\
insert (AMS align environment | AMS align) & Insert an aligned equation \\
insert AMS align at [environment] &  \\
insert AMS flalign [environment] &  \\
insert (AMS gathered environment | AMS gather) &  \\
insert (AMS multline [environment]| multiline) &  \\
insert array [environment] &  \\
insert (cases [environment] | piecewise) &  \\
insert (aligned [environment] | align) &  \\
insert aligned at [environment] &  \\
insert gathered [environment] &  \\
insert split [environment] &  \\
\end{tabularx}


\end{document}